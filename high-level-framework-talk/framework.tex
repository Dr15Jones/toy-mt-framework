\documentclass{beamer}
\usepackage{xspace}
\setbeamertemplate{navigation symbols}{}
\useinnertheme{rounded}
\usefonttheme{structureitalicserif}
\setbeamertemplate{footline}[frame number]

\newcommand{\art}{\textbf{art}}
\newcommand{\classname}[1]{\texttt{#1}}
\newcommand{\code}[1]{\texttt{#1}}
\newcommand{\cpp}{C\kern-0.15ex{+}\kern-0.1ex{+}\xspace}

\begin{document}

\begin{frame}[fragile,plain]
  \color{structure}
  \vskip .8in
  \begin{tabular}{lr}
    {\Large\itshape \parbox[c]{\textwidth}{HEP Event Processing Frameworks}} \\
    {\large\itshape \parbox[c]{\textwidth}{A very high-level view}}
  \end{tabular}
  \vskip .5in
  \begin{tabular}{lr}
    { \large Marc Paterno} &   \\
    { \large Christoper D. Jones } \\
    { \large 9 February 2012}       & \\
  \end{tabular}
\end{frame}

\begin{frame}
  \frametitle{Organization of this talk}
  \begin{itemize}
    \item The topics we cover are very inter-connected
    \item We have tried to impose an organization
    \item This organization makes sense to the authors; we hope it makes
          sense to you
  \end{itemize}
\end{frame}

\begin{frame}
  \frametitle{Facts of the problem domain}
  \begin{itemize}
    \item Our units of data processing, \emph{events}, are
          independent
    \item Event data are organized hierarchically: run/subrun/event
    \item Different levels of hierarchy represent different \emph{intervals
          of validity}
    \item Events are collections of tagged arbitrary \emph{structured} data
    \begin{itemize}
      \item new types are defined by physicists (not framework developers)
      \item frameworks don't know about the many types
    \end{itemize}
    \item Objects put into the event become immutable
    \item All events passing each of several possible physicist-defined
          \emph{filtering criteria} must be written to the same file
          (by the end of the job, they should be in the same file)
     \item Non-event \emph{summary data} (\textit{e.g.} histograms) are
           accumulated and stored in (possibly different) files
  \end{itemize}
\end{frame}

\begin{frame}
  \frametitle{Facts of the data flow and program organization}
  \begin{itemize}
    \item Framework programs are configured by physicists
          at \emph{runtime} to use
          arbitrary modules
    \item Modules typically contain physicist-specified
          runtime configuration parameters
    \item Module processing times range from milliseconds to many seconds
    \item Modules depend on data from other modules, but this dependency
          can change from event to event
    \item Modules communicate through only a few known channels (event,
          subrun, run); they \emph{never} call another module directly
    \item Frameworks can be configured to call some modules only if event
          data meet some \emph{filtering condition}
    \item Processed events are stored in external files, because
    \begin{itemize}
      \item processing is often slow
      \item we need reproducibility of and agreement on the definition of
            data samples
    \end{itemize}
  \end{itemize}
\end{frame}

\begin{frame}
  \frametitle{Facts of our development community}
  \begin{itemize}
    \item Modules are written in \cpp
    \item Module development is done independently by physicists
    \item We want physicists to be able to use multithreading \emph{within}
          a module (\textit{e.g.} OpenMP, libdispatch)
    \item Physicists mostly do not know how to write thread-safe code;
          we need to give them guidelines
    \begin{itemize}
      \item allow instances of distinct classes to be used simultaneously
      \item allow distinct instances of a single module class to be active
            in different threads
      \item allow one instance of a module to be used simultaneously from
            multiple threads
    \end{itemize}
    \item Physics frequently use libraries that are not thread-safe and that
          we do not control and can not (or will not) do without; we need
          to provide facilities and guidelines for their safe use
  \end{itemize}
\end{frame}

\end{document}
